\documentclass[pldi.tex]{subfiles}
\begin{document}
\section{On the relation with Brzozowski's Derivative}

Valiant's $\otimes$ operator, which solves for the set of productions unifying known factors in a binary CFG, implies the existence of a left- and right-quotient, which yield the set of nonterminals that may appear to the right- or left-side, respectively, of a known factor in a binary production. In other words, a known factor not only constrains subsequent expressions that can be derived from it, but also adjacent factors it may be composed with to form a new derivation. A more complete understanding of the solving process may be attained by considering the following three cases.

\begin{table}[H]
\begin{tabular}{ccccc}
    Valiant's $\otimes$ && Left Quotient && Right Quotient \\\\
    $a \otimes b = \{c \mid (c \rightarrow a b) \in P\}$ &&
    $\frac{\partial \Gamma}{\partial a} = \{b \mid (c \rightarrow a b) \in P\}$ &&
    $\frac{\laitrap \Gamma}{\laitrap b} = \{a \mid (c \rightarrow a b) \in P\}$ \\\\
    \begin{tabular}{|c|c|}
        \hline
        \cellcolor{black!15}a & c \\ \hline
        \multicolumn{1}{c|}{~} & \cellcolor{black!15}b \\
        \cline{2-2}
    \end{tabular} &&
    \begin{tabular}{|c|c|}
        \hline
        \cellcolor{black!15}a & c \\ \hline
        \multicolumn{1}{c|}{~} & b \\
        \cline{2-2}
    \end{tabular} &&
    \begin{tabular}{|c|c|}
        \hline
        a & c \\ \hline
        \multicolumn{1}{c|}{~} & \cellcolor{black!15}b \\
        \cline{2-2}
    \end{tabular}
\end{tabular}
\end{table}

The left quotient operator coincides with the derivative operator in the context-free setting originally considered by Brzozowski~\cite{brzozowski1964derivatives} and Antimirov~\cite{antimirov1996partial} and the right quotient corresponds to their inverse.

\end{document}
